


% Header, overrides base

    % Make sure that the sphinx doc style knows who it inherits from.
    \def\sphinxdocclass{article}

    % Declare the document class
    \documentclass[letterpaper,10pt,english]{/usr/share/sphinx/texinputs/sphinxhowto}

    % Imports
    \usepackage[utf8]{inputenc}
    \DeclareUnicodeCharacter{00A0}{\\nobreakspace}
    \usepackage[T1]{fontenc}
    \usepackage{babel}
    \usepackage{times}
    \usepackage{import}
    \usepackage[Bjarne]{/usr/share/sphinx/texinputs/fncychap}
    \usepackage{longtable}
    \usepackage{/usr/share/sphinx/texinputs/sphinx}
    \usepackage{multirow}

    \usepackage{amsmath}
    \usepackage{amssymb}
    \usepackage{ucs}
    \usepackage{enumerate}

    % Used to make the Input/Output rules follow around the contents.
    \usepackage{needspace}

    % Pygments requirements
    \usepackage{fancyvrb}
    \usepackage{color}
    % ansi colors additions
    \definecolor{darkgreen}{rgb}{.12,.54,.11}
    \definecolor{lightgray}{gray}{.95}
    \definecolor{brown}{rgb}{0.54,0.27,0.07}
    \definecolor{purple}{rgb}{0.5,0.0,0.5}
    \definecolor{darkgray}{gray}{0.25}
    \definecolor{lightred}{rgb}{1.0,0.39,0.28}
    \definecolor{lightgreen}{rgb}{0.48,0.99,0.0}
    \definecolor{lightblue}{rgb}{0.53,0.81,0.92}
    \definecolor{lightpurple}{rgb}{0.87,0.63,0.87}
    \definecolor{lightcyan}{rgb}{0.5,1.0,0.83}

    % Needed to box output/input
    \usepackage{tikz}
        \usetikzlibrary{calc,arrows,shadows}
    \usepackage[framemethod=tikz]{mdframed}

    \usepackage{alltt}

    % Used to load and display graphics
    \usepackage{graphicx}
    \graphicspath{ {figs/} }
    \usepackage[Export]{adjustbox} % To resize

    % used so that images for notebooks which have spaces in the name can still be included
    \usepackage{grffile}


    % For formatting output while also word wrapping.
    \usepackage{listings}
    \lstset{breaklines=true}
    \lstset{basicstyle=\small\ttfamily}
    \def\smaller{\fontsize{9.5pt}{9.5pt}\selectfont}

    %Pygments definitions
    
\makeatletter
\def\PY@reset{\let\PY@it=\relax \let\PY@bf=\relax%
    \let\PY@ul=\relax \let\PY@tc=\relax%
    \let\PY@bc=\relax \let\PY@ff=\relax}
\def\PY@tok#1{\csname PY@tok@#1\endcsname}
\def\PY@toks#1+{\ifx\relax#1\empty\else%
    \PY@tok{#1}\expandafter\PY@toks\fi}
\def\PY@do#1{\PY@bc{\PY@tc{\PY@ul{%
    \PY@it{\PY@bf{\PY@ff{#1}}}}}}}
\def\PY#1#2{\PY@reset\PY@toks#1+\relax+\PY@do{#2}}

\expandafter\def\csname PY@tok@gd\endcsname{\def\PY@tc##1{\textcolor[rgb]{0.63,0.00,0.00}{##1}}}
\expandafter\def\csname PY@tok@gu\endcsname{\let\PY@bf=\textbf\def\PY@tc##1{\textcolor[rgb]{0.50,0.00,0.50}{##1}}}
\expandafter\def\csname PY@tok@gt\endcsname{\def\PY@tc##1{\textcolor[rgb]{0.00,0.27,0.87}{##1}}}
\expandafter\def\csname PY@tok@gs\endcsname{\let\PY@bf=\textbf}
\expandafter\def\csname PY@tok@gr\endcsname{\def\PY@tc##1{\textcolor[rgb]{1.00,0.00,0.00}{##1}}}
\expandafter\def\csname PY@tok@cm\endcsname{\let\PY@it=\textit\def\PY@tc##1{\textcolor[rgb]{0.25,0.50,0.50}{##1}}}
\expandafter\def\csname PY@tok@vg\endcsname{\def\PY@tc##1{\textcolor[rgb]{0.10,0.09,0.49}{##1}}}
\expandafter\def\csname PY@tok@m\endcsname{\def\PY@tc##1{\textcolor[rgb]{0.40,0.40,0.40}{##1}}}
\expandafter\def\csname PY@tok@mh\endcsname{\def\PY@tc##1{\textcolor[rgb]{0.40,0.40,0.40}{##1}}}
\expandafter\def\csname PY@tok@go\endcsname{\def\PY@tc##1{\textcolor[rgb]{0.53,0.53,0.53}{##1}}}
\expandafter\def\csname PY@tok@ge\endcsname{\let\PY@it=\textit}
\expandafter\def\csname PY@tok@vc\endcsname{\def\PY@tc##1{\textcolor[rgb]{0.10,0.09,0.49}{##1}}}
\expandafter\def\csname PY@tok@il\endcsname{\def\PY@tc##1{\textcolor[rgb]{0.40,0.40,0.40}{##1}}}
\expandafter\def\csname PY@tok@cs\endcsname{\let\PY@it=\textit\def\PY@tc##1{\textcolor[rgb]{0.25,0.50,0.50}{##1}}}
\expandafter\def\csname PY@tok@cp\endcsname{\def\PY@tc##1{\textcolor[rgb]{0.74,0.48,0.00}{##1}}}
\expandafter\def\csname PY@tok@gi\endcsname{\def\PY@tc##1{\textcolor[rgb]{0.00,0.63,0.00}{##1}}}
\expandafter\def\csname PY@tok@gh\endcsname{\let\PY@bf=\textbf\def\PY@tc##1{\textcolor[rgb]{0.00,0.00,0.50}{##1}}}
\expandafter\def\csname PY@tok@ni\endcsname{\let\PY@bf=\textbf\def\PY@tc##1{\textcolor[rgb]{0.60,0.60,0.60}{##1}}}
\expandafter\def\csname PY@tok@nl\endcsname{\def\PY@tc##1{\textcolor[rgb]{0.63,0.63,0.00}{##1}}}
\expandafter\def\csname PY@tok@nn\endcsname{\let\PY@bf=\textbf\def\PY@tc##1{\textcolor[rgb]{0.00,0.00,1.00}{##1}}}
\expandafter\def\csname PY@tok@no\endcsname{\def\PY@tc##1{\textcolor[rgb]{0.53,0.00,0.00}{##1}}}
\expandafter\def\csname PY@tok@na\endcsname{\def\PY@tc##1{\textcolor[rgb]{0.49,0.56,0.16}{##1}}}
\expandafter\def\csname PY@tok@nb\endcsname{\def\PY@tc##1{\textcolor[rgb]{0.00,0.50,0.00}{##1}}}
\expandafter\def\csname PY@tok@nc\endcsname{\let\PY@bf=\textbf\def\PY@tc##1{\textcolor[rgb]{0.00,0.00,1.00}{##1}}}
\expandafter\def\csname PY@tok@nd\endcsname{\def\PY@tc##1{\textcolor[rgb]{0.67,0.13,1.00}{##1}}}
\expandafter\def\csname PY@tok@ne\endcsname{\let\PY@bf=\textbf\def\PY@tc##1{\textcolor[rgb]{0.82,0.25,0.23}{##1}}}
\expandafter\def\csname PY@tok@nf\endcsname{\def\PY@tc##1{\textcolor[rgb]{0.00,0.00,1.00}{##1}}}
\expandafter\def\csname PY@tok@si\endcsname{\let\PY@bf=\textbf\def\PY@tc##1{\textcolor[rgb]{0.73,0.40,0.53}{##1}}}
\expandafter\def\csname PY@tok@s2\endcsname{\def\PY@tc##1{\textcolor[rgb]{0.73,0.13,0.13}{##1}}}
\expandafter\def\csname PY@tok@vi\endcsname{\def\PY@tc##1{\textcolor[rgb]{0.10,0.09,0.49}{##1}}}
\expandafter\def\csname PY@tok@nt\endcsname{\let\PY@bf=\textbf\def\PY@tc##1{\textcolor[rgb]{0.00,0.50,0.00}{##1}}}
\expandafter\def\csname PY@tok@nv\endcsname{\def\PY@tc##1{\textcolor[rgb]{0.10,0.09,0.49}{##1}}}
\expandafter\def\csname PY@tok@s1\endcsname{\def\PY@tc##1{\textcolor[rgb]{0.73,0.13,0.13}{##1}}}
\expandafter\def\csname PY@tok@sh\endcsname{\def\PY@tc##1{\textcolor[rgb]{0.73,0.13,0.13}{##1}}}
\expandafter\def\csname PY@tok@sc\endcsname{\def\PY@tc##1{\textcolor[rgb]{0.73,0.13,0.13}{##1}}}
\expandafter\def\csname PY@tok@sx\endcsname{\def\PY@tc##1{\textcolor[rgb]{0.00,0.50,0.00}{##1}}}
\expandafter\def\csname PY@tok@bp\endcsname{\def\PY@tc##1{\textcolor[rgb]{0.00,0.50,0.00}{##1}}}
\expandafter\def\csname PY@tok@c1\endcsname{\let\PY@it=\textit\def\PY@tc##1{\textcolor[rgb]{0.25,0.50,0.50}{##1}}}
\expandafter\def\csname PY@tok@kc\endcsname{\let\PY@bf=\textbf\def\PY@tc##1{\textcolor[rgb]{0.00,0.50,0.00}{##1}}}
\expandafter\def\csname PY@tok@c\endcsname{\let\PY@it=\textit\def\PY@tc##1{\textcolor[rgb]{0.25,0.50,0.50}{##1}}}
\expandafter\def\csname PY@tok@mf\endcsname{\def\PY@tc##1{\textcolor[rgb]{0.40,0.40,0.40}{##1}}}
\expandafter\def\csname PY@tok@err\endcsname{\def\PY@bc##1{\setlength{\fboxsep}{0pt}\fcolorbox[rgb]{1.00,0.00,0.00}{1,1,1}{\strut ##1}}}
\expandafter\def\csname PY@tok@kd\endcsname{\let\PY@bf=\textbf\def\PY@tc##1{\textcolor[rgb]{0.00,0.50,0.00}{##1}}}
\expandafter\def\csname PY@tok@ss\endcsname{\def\PY@tc##1{\textcolor[rgb]{0.10,0.09,0.49}{##1}}}
\expandafter\def\csname PY@tok@sr\endcsname{\def\PY@tc##1{\textcolor[rgb]{0.73,0.40,0.53}{##1}}}
\expandafter\def\csname PY@tok@mo\endcsname{\def\PY@tc##1{\textcolor[rgb]{0.40,0.40,0.40}{##1}}}
\expandafter\def\csname PY@tok@kn\endcsname{\let\PY@bf=\textbf\def\PY@tc##1{\textcolor[rgb]{0.00,0.50,0.00}{##1}}}
\expandafter\def\csname PY@tok@mi\endcsname{\def\PY@tc##1{\textcolor[rgb]{0.40,0.40,0.40}{##1}}}
\expandafter\def\csname PY@tok@gp\endcsname{\let\PY@bf=\textbf\def\PY@tc##1{\textcolor[rgb]{0.00,0.00,0.50}{##1}}}
\expandafter\def\csname PY@tok@o\endcsname{\def\PY@tc##1{\textcolor[rgb]{0.40,0.40,0.40}{##1}}}
\expandafter\def\csname PY@tok@kr\endcsname{\let\PY@bf=\textbf\def\PY@tc##1{\textcolor[rgb]{0.00,0.50,0.00}{##1}}}
\expandafter\def\csname PY@tok@s\endcsname{\def\PY@tc##1{\textcolor[rgb]{0.73,0.13,0.13}{##1}}}
\expandafter\def\csname PY@tok@kp\endcsname{\def\PY@tc##1{\textcolor[rgb]{0.00,0.50,0.00}{##1}}}
\expandafter\def\csname PY@tok@w\endcsname{\def\PY@tc##1{\textcolor[rgb]{0.73,0.73,0.73}{##1}}}
\expandafter\def\csname PY@tok@kt\endcsname{\def\PY@tc##1{\textcolor[rgb]{0.69,0.00,0.25}{##1}}}
\expandafter\def\csname PY@tok@ow\endcsname{\let\PY@bf=\textbf\def\PY@tc##1{\textcolor[rgb]{0.67,0.13,1.00}{##1}}}
\expandafter\def\csname PY@tok@sb\endcsname{\def\PY@tc##1{\textcolor[rgb]{0.73,0.13,0.13}{##1}}}
\expandafter\def\csname PY@tok@k\endcsname{\let\PY@bf=\textbf\def\PY@tc##1{\textcolor[rgb]{0.00,0.50,0.00}{##1}}}
\expandafter\def\csname PY@tok@se\endcsname{\let\PY@bf=\textbf\def\PY@tc##1{\textcolor[rgb]{0.73,0.40,0.13}{##1}}}
\expandafter\def\csname PY@tok@sd\endcsname{\let\PY@it=\textit\def\PY@tc##1{\textcolor[rgb]{0.73,0.13,0.13}{##1}}}

\def\PYZbs{\char`\\}
\def\PYZus{\char`\_}
\def\PYZob{\char`\{}
\def\PYZcb{\char`\}}
\def\PYZca{\char`\^}
\def\PYZam{\char`\&}
\def\PYZlt{\char`\<}
\def\PYZgt{\char`\>}
\def\PYZsh{\char`\#}
\def\PYZpc{\char`\%}
\def\PYZdl{\char`\$}
\def\PYZhy{\char`\-}
\def\PYZsq{\char`\'}
\def\PYZdq{\char`\"}
\def\PYZti{\char`\~}
% for compatibility with earlier versions
\def\PYZat{@}
\def\PYZlb{[}
\def\PYZrb{]}
\makeatother


    %Set pygments styles if needed...
    
        \definecolor{nbframe-border}{rgb}{0.867,0.867,0.867}
        \definecolor{nbframe-bg}{rgb}{0.969,0.969,0.969}
        \definecolor{nbframe-in-prompt}{rgb}{0.0,0.0,0.502}
        \definecolor{nbframe-out-prompt}{rgb}{0.545,0.0,0.0}

        \newenvironment{ColorVerbatim}
        {\begin{mdframed}[%
            roundcorner=1.0pt, %
            backgroundcolor=nbframe-bg, %
            userdefinedwidth=1\linewidth, %
            leftmargin=0.1\linewidth, %
            innerleftmargin=0pt, %
            innerrightmargin=0pt, %
            linecolor=nbframe-border, %
            linewidth=1pt, %
            usetwoside=false, %
            everyline=true, %
            innerlinewidth=3pt, %
            innerlinecolor=nbframe-bg, %
            middlelinewidth=1pt, %
            middlelinecolor=nbframe-bg, %
            outerlinewidth=0.5pt, %
            outerlinecolor=nbframe-border, %
            needspace=0pt
        ]}
        {\end{mdframed}}
        
        \newenvironment{InvisibleVerbatim}
        {\begin{mdframed}[leftmargin=0.1\linewidth,innerleftmargin=3pt,innerrightmargin=3pt, userdefinedwidth=1\linewidth, linewidth=0pt, linecolor=white, usetwoside=false]}
        {\end{mdframed}}

        \renewenvironment{Verbatim}[1][\unskip]
        {\begin{alltt}\smaller}
        {\end{alltt}}
    

    % Help prevent overflowing lines due to urls and other hard-to-break 
    % entities.  This doesn't catch everything...
    \sloppy

    % Document level variables
    \title{NonlinearDiffusionEquation }
    \date{December 2, 2013}
    \release{}
    \author{Milad H. Mobarhan}
    \renewcommand{\releasename}{}

    % TODO: Add option for the user to specify a logo for his/her export.
    \newcommand{\sphinxlogo}{}

    % Make the index page of the document.
    \makeindex

    % Import sphinx document type specifics.
     


% Body

    % Start of the document
    \begin{document}

        
            \maketitle
        

        


        
        \section{Project 3: Nonlinear diffusion equation}

\emph{Summary.} The goal of this project is to discuss various numerical
aspects of a nonlinear diffusion model.\begin{center}\rule{3in}{0.4pt}\end{center}

\subsection{Mathematical problem}

We look at the PDE problem

\begin{align*}
\varrho u_t &= \nabla\cdot (\alpha (u)\nabla u) + f(\pmb{x}, t),\quad &t >0\\
u(\pmb{x},0) &= I(\pmb{x}),\quad &\pmb{x}\in\Omega\\
\alpha (u(\pmb{x},t))\frac{\partial}{\partial n} u(\pmb{x}, t) &= 0,\quad &\pmb{x}\in\partial\Omega_N
\end{align*}

The coefficiet $\varrho$ is constant and $\alpha (u)$ is a known
function of $u$.\begin{center}\rule{3in}{0.4pt}\end{center}

\subsection{Discretization in time}

First we need a mesh in time, here taken as uniform with mesh points
$t_n=n \Delta t$, $n=0,1,…,N_t$. A Backward Euler scheme consists of
sampling at $t_n$ and approximating the time derivative by a backward
difference

\[[D_t^- u]^n\approx (u^{n}-u^{n-1})/\Delta t.\]

This approximation turns our PDE into a differential equation that is
discrete in time, but still continuous in space. With a finite
difference operator notation we can write the time-discrete problem as

\[
[D_t^- \varrho u = \nabla\cdot (\alpha (u)\nabla u) + f(\boldsymbol{x}, t)]^n
{\thinspace .}
\]

which gives the nonlinear time-discrete PDEs

\[
u^{n} - \frac{\Delta t}{\varrho} \left[ \nabla\cdot (\alpha (u)\nabla u^n) + f(\boldsymbol{x}, t_{n})\right] = {u}^{n-1}
\]

or with $u^n = u$ and $u^{n-1} = u_1$:

\[
u - \frac{\Delta t}{\varrho}\nabla\cdot({\alpha}(u^n)\nabla u) -\frac{\Delta t}{\varrho} f(\boldsymbol{x}, t_{n}) = u_1{\thinspace .}
\]

From the last equation we can define the residual:

\[
R = u - \frac{\Delta t}{\varrho}\nabla\cdot({\alpha}(u)\nabla u) -\frac{\Delta t}{\varrho} f(\boldsymbol{x}, t_{n}) - u_1{\thinspace .}
\]\begin{center}\rule{3in}{0.4pt}\end{center}

\subsection{The variational form}The least-squares principle is equivalent to demanding the error to be
orthogonal to the space $V$ when approximating a function $f$ by
$u \in V$. With a differential equation we do not know the true error so
we must instead require the residual $R$ to be orthogonal to $V$. This
idea implies seeking $\left\{ {c}_i \right\}_{i\in{\mathcal{I}_s}}$ such
that

\[
(R,v)=0,\quad \forall v\in V{\thinspace .}
\]

This statement is equivalent to $R$ being orthogonal to the $N+1$ basis
functions only:

\[
(R,{\psi}_i)=0,\quad i\in{\mathcal{I}_s},
\]

resulting in $N+1$ equations for determining
$\left\{ {c}_i \right\}_{i\in{\mathcal{I}_s}}$. The variational form for
our specific case is given by

\begin{align*}
\int_\Omega (u{\psi}_i - \frac{\Delta t}{\varrho}\nabla\cdot({\alpha}(u^n)\nabla u) {\psi}_i  - \frac{\Delta t}{\varrho} f(\boldsymbol{x}, t_{n}){\psi}_i - u_1{\psi}_i){\, \mathrm{d}x} &= 0\\
\int_\Omega (u{\psi}_i - \frac{\Delta t}{\varrho} \left[
-\int_{\Omega}{\alpha(u)}\nabla u\cdot\nabla{\psi}_i{\, \mathrm{d}x} +
\int_{\partial\Omega}{\alpha(u)}\frac{\partial u}{\partial n}{\psi}_i {\, \mathrm{d}x} \right]
- \frac{\Delta t}{\varrho} f(\boldsymbol{x}, t_{n}){\psi}_i - u_1{\psi}_i){\, \mathrm{d}x} &= 0\\
\int_\Omega (u{\psi}_i + \frac{\Delta t}{\varrho}{\alpha}(u)\nabla u\cdot\nabla {\psi}_i  - \frac{\Delta t}{\varrho} f(\boldsymbol{x}, t_{n}){\psi}_i - u_1{\psi}_i){\, \mathrm{d}x} &= 0
\end{align*}

or more compactly

\[
(u,{\psi}_i) + \frac{\Delta t}{\varrho} ({\alpha}\nabla u,\nabla{\psi}_i)
= (u_1{\psi}_i) + \frac{\Delta t}{\varrho} (f^n,{\psi}_i)
\]

\subsubsection{Initial condition}

The variational form for the initial condition is found by expanding

\[
u(\pmb{x},0) = \sum_j c_j^0 \psi_j(\pmb{x}) = I(\pmb{x})
\]

The Galerkin method implies

\[
\left(\sum_j c_j^0 \psi_j(\pmb{x}) - I(\pmb{x}),{\psi}_i \right)=0,\quad i\in{\mathcal{I}_s},
\]

or

\[
\sum_j (\psi_j, {\psi}_i) c_j^0 =  (I,{\psi}_i)
\]\begin{center}\rule{3in}{0.4pt}\end{center}

\subsection{Picard iteration method at the PDE level}

Our aim is to discretize the problem in time and then present techniques
for linearizing the time-discrete PDE problem ``at the PDE level'' such
that we transform the nonlinear stationary PDE problems at each time
level into a sequence of linear PDE problems, which can be solved using
any method for linear PDEs. In our case we have

\[
u = \frac{\Delta t}{\varrho}\nabla\cdot({\alpha}(u)\nabla u) +\frac{\Delta t}{\varrho} f(\boldsymbol{x}, t_{n}) + u_1
\]

which is nonlinear beacuse of the dependency on $u$ in the variable
coefficient $\alpha$. Picard iteration needs a linearization where we
use the most recent approximation $u_-$ to $u$ in $\alpha$:

\[
u = -\frac{\Delta t}{\varrho}\nabla\cdot({\alpha}(u_-)\nabla u) +\frac{\Delta t}{\varrho} f(\boldsymbol{x}, t_{n}) + u_1
\]

In variational form is given by

\[
(u,{\psi}_i) + \frac{\Delta t}{\varrho} ({\alpha(u_-)}\nabla u,\nabla{\psi}_i)
= (u_1{\psi}_i) + \frac{\Delta t}{\varrho} (f^n,{\psi}_i)
\]\begin{center}\rule{3in}{0.4pt}\end{center}

\subsection{Convergence test}

The first verification of the FEniCS implementation may assmue
$\alpha (u)=1$, $f=0$, $\Omega = [0,1]\times [0,1]$, P1 elements, and
$I(x,y)=\cos(\pi x)$. The exact solution is then
$u(x,y,t)=e^{-\pi^2 t}\cos (\pi x)$. The error in space is then
${\cal O}(\Delta x^2) + {\cal O}(\Delta y^2)$, while the error in time
is ${\cal O}(\Delta t^p)$, with $p=1$ for the Backward Euler scheme and
$p=2$ for the Crank-Nicolson or the 2-step backward schemes. We set
$h=\Delta t^p = \Delta x^2$ and perform a convergence rate study as $h$
is decreased, using Backward Euler scheme:

\begin{verbatim}
h =  0.1       E =  0.000497030720246  r =  1.22455181028
h =  0.05      E =  0.000212694571666  r =  1.66702631777
h =  0.025     E =  6.69778929328e-05  r =  1.60581512224
h =  0.0125    E =  2.20055871886e-05  r =  1.09776873673
h =  0.00625   E =  1.02818589919e-05  r =  1.03661750021
h =  0.003125  E =  5.01208793551e-06  r =  1.0386373625
\end{verbatim}

where $r = E/h$, which is approximately constant, as expected.\begin{center}\rule{3in}{0.4pt}\end{center}

\subsection{Manufactored solution}

To get an indication whether the implementation of the nonlinear
diffusion PDE is correct or not, we can use the method of manufactured
solutions. Say we restrict the problem to one space dimension,
$\Omega=[0,1]$, and choose

\[
u(x,t) = t\int_0^x q(1-q)dq = tx^2\left(\frac{1}{2} - \frac{x}{3}\right)
\]

and $\alpha(u) = 1+u^2$. The following sympy session computes an
$f(x,t)$ such that the above $u$ is a solution of the PDE problem:

    % Make sure that atleast 4 lines are below the HR
    \needspace{4\baselineskip}

    
        \vspace{6pt}
        \makebox[0.1\linewidth]{\smaller\hfill\tt\color{nbframe-in-prompt}In\hspace{4pt}{[}36{]}:\hspace{4pt}}\\*
        \vspace{-2.65\baselineskip}
        \begin{ColorVerbatim}
            \vspace{-0.7\baselineskip}
            \begin{Verbatim}[commandchars=\\\{\}]
\PY{k+kn}{from} \PY{n+nn}{sympy} \PY{k+kn}{import} \PY{o}{*}

\PY{k}{def} \PY{n+nf}{a}\PY{p}{(}\PY{n}{u}\PY{p}{)}\PY{p}{:}
    \PY{k}{return} \PY{l+m+mi}{1} \PY{o}{+} \PY{n}{u}\PY{o}{*}\PY{o}{*}\PY{l+m+mi}{2}

\PY{k}{def} \PY{n+nf}{u\PYZus{}simple}\PY{p}{(}\PY{n}{x}\PY{p}{,} \PY{n}{t}\PY{p}{)}\PY{p}{:}
    \PY{k}{return} \PY{n}{x}\PY{o}{*}\PY{o}{*}\PY{l+m+mi}{2}\PY{o}{*}\PY{p}{(}\PY{n}{Rational}\PY{p}{(}\PY{l+m+mi}{1}\PY{p}{,}\PY{l+m+mi}{2}\PY{p}{)} \PY{o}{\PYZhy{}} \PY{n}{x}\PY{o}{/}\PY{l+m+mi}{3}\PY{p}{)}\PY{o}{*}\PY{n}{t}

\PY{n}{x}\PY{p}{,} \PY{n}{t}\PY{p}{,} \PY{n}{p}\PY{p}{,} \PY{n}{dt} \PY{o}{=} \PY{n}{symbols}\PY{p}{(}\PY{l+s}{\PYZsq{}}\PY{l+s}{x[0] t p dt}\PY{l+s}{\PYZsq{}}\PY{p}{)}
\PY{k}{for} \PY{n}{x\PYZus{}point} \PY{o+ow}{in} \PY{l+m+mi}{0}\PY{p}{,} \PY{l+m+mi}{1}\PY{p}{:}
     \PY{k}{print} \PY{l+s}{\PYZsq{}}\PY{l+s}{u\PYZus{}x(}\PY{l+s+si}{\PYZpc{}s}\PY{l+s}{,t):}\PY{l+s}{\PYZsq{}} \PY{o}{\PYZpc{}} \PY{n}{x\PYZus{}point}\PY{p}{,}
     \PY{k}{print} \PY{n}{diff}\PY{p}{(}\PY{n}{u\PYZus{}simple}\PY{p}{(}\PY{n}{x}\PY{p}{,} \PY{n}{t}\PY{p}{)}\PY{p}{,} \PY{n}{x}\PY{p}{)}\PY{o}{.}\PY{n}{subs}\PY{p}{(}\PY{n}{x}\PY{p}{,} \PY{n}{x\PYZus{}point}\PY{p}{)}\PY{o}{.}\PY{n}{simplify}\PY{p}{(}\PY{p}{)}

\PY{k}{print} \PY{l+s}{\PYZsq{}}\PY{l+s}{Initial condition:}\PY{l+s}{\PYZsq{}}\PY{p}{,} \PY{n}{u\PYZus{}simple}\PY{p}{(}\PY{n}{x}\PY{p}{,} \PY{l+m+mi}{0}\PY{p}{)}

\PY{n}{u} \PY{o}{=} \PY{n}{u\PYZus{}simple}\PY{p}{(}\PY{n}{x}\PY{p}{,} \PY{n}{t}\PY{p}{)}
\PY{n}{f} \PY{o}{=} \PY{n}{p}\PY{o}{*}\PY{n}{diff}\PY{p}{(}\PY{n}{u}\PY{p}{,} \PY{n}{t}\PY{p}{)} \PY{o}{\PYZhy{}} \PY{n}{diff}\PY{p}{(}\PY{n}{a}\PY{p}{(}\PY{n}{u}\PY{p}{)}\PY{o}{*}\PY{n}{diff}\PY{p}{(}\PY{n}{u}\PY{p}{,} \PY{n}{x}\PY{p}{)}\PY{p}{,} \PY{n}{x}\PY{p}{)}
\PY{k}{print} \PY{n}{ccode}\PY{p}{(}\PY{n}{f}\PY{o}{.}\PY{n}{simplify}\PY{p}{(}\PY{p}{)}\PY{p}{)}
\end{Verbatim}

            
                \vspace{-0.2\baselineskip}
            
        \end{ColorVerbatim}
    

    

        % If the first block is an image, minipage the image.  Else
        % request a certain amount of space for the input text.
        \needspace{4\baselineskip}
        
        

            % Add document contents.
            
                \begin{InvisibleVerbatim}
                \vspace{-0.5\baselineskip}
\begin{alltt}u\_x(0,t): 0
u\_x(1,t): 0
Initial condition: 0
-p*pow(x[0], 3)/3 + p*pow(x[0], 2)/2 + 8*pow(t, 3)*pow(x[0], 7)/9 -
28*pow(t, 3)*pow(x[0], 6)/9 + 7*pow(t, 3)*pow(x[0], 5)/2 - 5*pow(t,
3)*pow(x[0], 4)/4 + 2*t*x[0] - t
\end{alltt}

            \end{InvisibleVerbatim}
            
        
    
We get the following results for different final times:

\begin{verbatim}
T =  0.1  E =  3.7822919294e-08
T =  0.5  E =  1.86763505521e-06
T =  1.0  E =  9.02389336352e-06
T =  2.0  E =  4.29791921834e-05
T =  3.0  E =  0.000131418978921
\end{verbatim}

For much higher values of $T$, the error become more significant, which
is due to the error in the single Picard iteration.\begin{center}\rule{3in}{0.4pt}\end{center}

\subsection{Sources of numerical errors}

\begin{itemize}
\item
  Truncation error from approximating the function with piecewise linear
  elements.
\item
  Error due to a single Picard iteration
\item
  Time discretisation (not explicitly due to the FEniCS)
\end{itemize}
        

        \renewcommand{\indexname}{Index}
        \printindex

    % End of document
    \end{document}


