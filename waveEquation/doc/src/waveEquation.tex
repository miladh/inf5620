


% Header, overrides base

    % Make sure that the sphinx doc style knows who it inherits from.
    \def\sphinxdocclass{article}

    % Declare the document class
    \documentclass[letterpaper,10pt,english]{/usr/share/sphinx/texinputs/sphinxhowto}

    % Imports
    \usepackage[utf8]{inputenc}
    \DeclareUnicodeCharacter{00A0}{\\nobreakspace}
    \usepackage[T1]{fontenc}
    \usepackage{babel}
    \usepackage{times}
    \usepackage{import}
    \usepackage[Bjarne]{/usr/share/sphinx/texinputs/fncychap}
    \usepackage{longtable}
    \usepackage{/usr/share/sphinx/texinputs/sphinx}
    \usepackage{multirow}

    \usepackage{amsmath}
    \usepackage{amssymb}
    \usepackage{ucs}
    \usepackage{enumerate}

    % Used to make the Input/Output rules follow around the contents.
    \usepackage{needspace}

    % Pygments requirements
    \usepackage{fancyvrb}
    \usepackage{color}
    % ansi colors additions
    \definecolor{darkgreen}{rgb}{.12,.54,.11}
    \definecolor{lightgray}{gray}{.95}
    \definecolor{brown}{rgb}{0.54,0.27,0.07}
    \definecolor{purple}{rgb}{0.5,0.0,0.5}
    \definecolor{darkgray}{gray}{0.25}
    \definecolor{lightred}{rgb}{1.0,0.39,0.28}
    \definecolor{lightgreen}{rgb}{0.48,0.99,0.0}
    \definecolor{lightblue}{rgb}{0.53,0.81,0.92}
    \definecolor{lightpurple}{rgb}{0.87,0.63,0.87}
    \definecolor{lightcyan}{rgb}{0.5,1.0,0.83}

    % Needed to box output/input
    \usepackage{tikz}
        \usetikzlibrary{calc,arrows,shadows}
    \usepackage[framemethod=tikz]{mdframed}

    \usepackage{alltt}

    % Used to load and display graphics
    \usepackage{graphicx}
    \graphicspath{ {figs/} }
    \usepackage[Export]{adjustbox} % To resize

    % used so that images for notebooks which have spaces in the name can still be included
    \usepackage{grffile}


    % For formatting output while also word wrapping.
    \usepackage{listings}
    \lstset{breaklines=true}
    \lstset{basicstyle=\small\ttfamily}
    \def\smaller{\fontsize{9.5pt}{9.5pt}\selectfont}

    %Pygments definitions
    
\makeatletter
\def\PY@reset{\let\PY@it=\relax \let\PY@bf=\relax%
    \let\PY@ul=\relax \let\PY@tc=\relax%
    \let\PY@bc=\relax \let\PY@ff=\relax}
\def\PY@tok#1{\csname PY@tok@#1\endcsname}
\def\PY@toks#1+{\ifx\relax#1\empty\else%
    \PY@tok{#1}\expandafter\PY@toks\fi}
\def\PY@do#1{\PY@bc{\PY@tc{\PY@ul{%
    \PY@it{\PY@bf{\PY@ff{#1}}}}}}}
\def\PY#1#2{\PY@reset\PY@toks#1+\relax+\PY@do{#2}}

\expandafter\def\csname PY@tok@gd\endcsname{\def\PY@tc##1{\textcolor[rgb]{0.63,0.00,0.00}{##1}}}
\expandafter\def\csname PY@tok@gu\endcsname{\let\PY@bf=\textbf\def\PY@tc##1{\textcolor[rgb]{0.50,0.00,0.50}{##1}}}
\expandafter\def\csname PY@tok@gt\endcsname{\def\PY@tc##1{\textcolor[rgb]{0.00,0.27,0.87}{##1}}}
\expandafter\def\csname PY@tok@gs\endcsname{\let\PY@bf=\textbf}
\expandafter\def\csname PY@tok@gr\endcsname{\def\PY@tc##1{\textcolor[rgb]{1.00,0.00,0.00}{##1}}}
\expandafter\def\csname PY@tok@cm\endcsname{\let\PY@it=\textit\def\PY@tc##1{\textcolor[rgb]{0.25,0.50,0.50}{##1}}}
\expandafter\def\csname PY@tok@vg\endcsname{\def\PY@tc##1{\textcolor[rgb]{0.10,0.09,0.49}{##1}}}
\expandafter\def\csname PY@tok@m\endcsname{\def\PY@tc##1{\textcolor[rgb]{0.40,0.40,0.40}{##1}}}
\expandafter\def\csname PY@tok@mh\endcsname{\def\PY@tc##1{\textcolor[rgb]{0.40,0.40,0.40}{##1}}}
\expandafter\def\csname PY@tok@go\endcsname{\def\PY@tc##1{\textcolor[rgb]{0.53,0.53,0.53}{##1}}}
\expandafter\def\csname PY@tok@ge\endcsname{\let\PY@it=\textit}
\expandafter\def\csname PY@tok@vc\endcsname{\def\PY@tc##1{\textcolor[rgb]{0.10,0.09,0.49}{##1}}}
\expandafter\def\csname PY@tok@il\endcsname{\def\PY@tc##1{\textcolor[rgb]{0.40,0.40,0.40}{##1}}}
\expandafter\def\csname PY@tok@cs\endcsname{\let\PY@it=\textit\def\PY@tc##1{\textcolor[rgb]{0.25,0.50,0.50}{##1}}}
\expandafter\def\csname PY@tok@cp\endcsname{\def\PY@tc##1{\textcolor[rgb]{0.74,0.48,0.00}{##1}}}
\expandafter\def\csname PY@tok@gi\endcsname{\def\PY@tc##1{\textcolor[rgb]{0.00,0.63,0.00}{##1}}}
\expandafter\def\csname PY@tok@gh\endcsname{\let\PY@bf=\textbf\def\PY@tc##1{\textcolor[rgb]{0.00,0.00,0.50}{##1}}}
\expandafter\def\csname PY@tok@ni\endcsname{\let\PY@bf=\textbf\def\PY@tc##1{\textcolor[rgb]{0.60,0.60,0.60}{##1}}}
\expandafter\def\csname PY@tok@nl\endcsname{\def\PY@tc##1{\textcolor[rgb]{0.63,0.63,0.00}{##1}}}
\expandafter\def\csname PY@tok@nn\endcsname{\let\PY@bf=\textbf\def\PY@tc##1{\textcolor[rgb]{0.00,0.00,1.00}{##1}}}
\expandafter\def\csname PY@tok@no\endcsname{\def\PY@tc##1{\textcolor[rgb]{0.53,0.00,0.00}{##1}}}
\expandafter\def\csname PY@tok@na\endcsname{\def\PY@tc##1{\textcolor[rgb]{0.49,0.56,0.16}{##1}}}
\expandafter\def\csname PY@tok@nb\endcsname{\def\PY@tc##1{\textcolor[rgb]{0.00,0.50,0.00}{##1}}}
\expandafter\def\csname PY@tok@nc\endcsname{\let\PY@bf=\textbf\def\PY@tc##1{\textcolor[rgb]{0.00,0.00,1.00}{##1}}}
\expandafter\def\csname PY@tok@nd\endcsname{\def\PY@tc##1{\textcolor[rgb]{0.67,0.13,1.00}{##1}}}
\expandafter\def\csname PY@tok@ne\endcsname{\let\PY@bf=\textbf\def\PY@tc##1{\textcolor[rgb]{0.82,0.25,0.23}{##1}}}
\expandafter\def\csname PY@tok@nf\endcsname{\def\PY@tc##1{\textcolor[rgb]{0.00,0.00,1.00}{##1}}}
\expandafter\def\csname PY@tok@si\endcsname{\let\PY@bf=\textbf\def\PY@tc##1{\textcolor[rgb]{0.73,0.40,0.53}{##1}}}
\expandafter\def\csname PY@tok@s2\endcsname{\def\PY@tc##1{\textcolor[rgb]{0.73,0.13,0.13}{##1}}}
\expandafter\def\csname PY@tok@vi\endcsname{\def\PY@tc##1{\textcolor[rgb]{0.10,0.09,0.49}{##1}}}
\expandafter\def\csname PY@tok@nt\endcsname{\let\PY@bf=\textbf\def\PY@tc##1{\textcolor[rgb]{0.00,0.50,0.00}{##1}}}
\expandafter\def\csname PY@tok@nv\endcsname{\def\PY@tc##1{\textcolor[rgb]{0.10,0.09,0.49}{##1}}}
\expandafter\def\csname PY@tok@s1\endcsname{\def\PY@tc##1{\textcolor[rgb]{0.73,0.13,0.13}{##1}}}
\expandafter\def\csname PY@tok@sh\endcsname{\def\PY@tc##1{\textcolor[rgb]{0.73,0.13,0.13}{##1}}}
\expandafter\def\csname PY@tok@sc\endcsname{\def\PY@tc##1{\textcolor[rgb]{0.73,0.13,0.13}{##1}}}
\expandafter\def\csname PY@tok@sx\endcsname{\def\PY@tc##1{\textcolor[rgb]{0.00,0.50,0.00}{##1}}}
\expandafter\def\csname PY@tok@bp\endcsname{\def\PY@tc##1{\textcolor[rgb]{0.00,0.50,0.00}{##1}}}
\expandafter\def\csname PY@tok@c1\endcsname{\let\PY@it=\textit\def\PY@tc##1{\textcolor[rgb]{0.25,0.50,0.50}{##1}}}
\expandafter\def\csname PY@tok@kc\endcsname{\let\PY@bf=\textbf\def\PY@tc##1{\textcolor[rgb]{0.00,0.50,0.00}{##1}}}
\expandafter\def\csname PY@tok@c\endcsname{\let\PY@it=\textit\def\PY@tc##1{\textcolor[rgb]{0.25,0.50,0.50}{##1}}}
\expandafter\def\csname PY@tok@mf\endcsname{\def\PY@tc##1{\textcolor[rgb]{0.40,0.40,0.40}{##1}}}
\expandafter\def\csname PY@tok@err\endcsname{\def\PY@bc##1{\setlength{\fboxsep}{0pt}\fcolorbox[rgb]{1.00,0.00,0.00}{1,1,1}{\strut ##1}}}
\expandafter\def\csname PY@tok@kd\endcsname{\let\PY@bf=\textbf\def\PY@tc##1{\textcolor[rgb]{0.00,0.50,0.00}{##1}}}
\expandafter\def\csname PY@tok@ss\endcsname{\def\PY@tc##1{\textcolor[rgb]{0.10,0.09,0.49}{##1}}}
\expandafter\def\csname PY@tok@sr\endcsname{\def\PY@tc##1{\textcolor[rgb]{0.73,0.40,0.53}{##1}}}
\expandafter\def\csname PY@tok@mo\endcsname{\def\PY@tc##1{\textcolor[rgb]{0.40,0.40,0.40}{##1}}}
\expandafter\def\csname PY@tok@kn\endcsname{\let\PY@bf=\textbf\def\PY@tc##1{\textcolor[rgb]{0.00,0.50,0.00}{##1}}}
\expandafter\def\csname PY@tok@mi\endcsname{\def\PY@tc##1{\textcolor[rgb]{0.40,0.40,0.40}{##1}}}
\expandafter\def\csname PY@tok@gp\endcsname{\let\PY@bf=\textbf\def\PY@tc##1{\textcolor[rgb]{0.00,0.00,0.50}{##1}}}
\expandafter\def\csname PY@tok@o\endcsname{\def\PY@tc##1{\textcolor[rgb]{0.40,0.40,0.40}{##1}}}
\expandafter\def\csname PY@tok@kr\endcsname{\let\PY@bf=\textbf\def\PY@tc##1{\textcolor[rgb]{0.00,0.50,0.00}{##1}}}
\expandafter\def\csname PY@tok@s\endcsname{\def\PY@tc##1{\textcolor[rgb]{0.73,0.13,0.13}{##1}}}
\expandafter\def\csname PY@tok@kp\endcsname{\def\PY@tc##1{\textcolor[rgb]{0.00,0.50,0.00}{##1}}}
\expandafter\def\csname PY@tok@w\endcsname{\def\PY@tc##1{\textcolor[rgb]{0.73,0.73,0.73}{##1}}}
\expandafter\def\csname PY@tok@kt\endcsname{\def\PY@tc##1{\textcolor[rgb]{0.69,0.00,0.25}{##1}}}
\expandafter\def\csname PY@tok@ow\endcsname{\let\PY@bf=\textbf\def\PY@tc##1{\textcolor[rgb]{0.67,0.13,1.00}{##1}}}
\expandafter\def\csname PY@tok@sb\endcsname{\def\PY@tc##1{\textcolor[rgb]{0.73,0.13,0.13}{##1}}}
\expandafter\def\csname PY@tok@k\endcsname{\let\PY@bf=\textbf\def\PY@tc##1{\textcolor[rgb]{0.00,0.50,0.00}{##1}}}
\expandafter\def\csname PY@tok@se\endcsname{\let\PY@bf=\textbf\def\PY@tc##1{\textcolor[rgb]{0.73,0.40,0.13}{##1}}}
\expandafter\def\csname PY@tok@sd\endcsname{\let\PY@it=\textit\def\PY@tc##1{\textcolor[rgb]{0.73,0.13,0.13}{##1}}}

\def\PYZbs{\char`\\}
\def\PYZus{\char`\_}
\def\PYZob{\char`\{}
\def\PYZcb{\char`\}}
\def\PYZca{\char`\^}
\def\PYZam{\char`\&}
\def\PYZlt{\char`\<}
\def\PYZgt{\char`\>}
\def\PYZsh{\char`\#}
\def\PYZpc{\char`\%}
\def\PYZdl{\char`\$}
\def\PYZhy{\char`\-}
\def\PYZsq{\char`\'}
\def\PYZdq{\char`\"}
\def\PYZti{\char`\~}
% for compatibility with earlier versions
\def\PYZat{@}
\def\PYZlb{[}
\def\PYZrb{]}
\makeatother


    %Set pygments styles if needed...
    
        \definecolor{nbframe-border}{rgb}{0.867,0.867,0.867}
        \definecolor{nbframe-bg}{rgb}{0.969,0.969,0.969}
        \definecolor{nbframe-in-prompt}{rgb}{0.0,0.0,0.502}
        \definecolor{nbframe-out-prompt}{rgb}{0.545,0.0,0.0}

        \newenvironment{ColorVerbatim}
        {\begin{mdframed}[%
            roundcorner=1.0pt, %
            backgroundcolor=nbframe-bg, %
            userdefinedwidth=1\linewidth, %
            leftmargin=0.1\linewidth, %
            innerleftmargin=0pt, %
            innerrightmargin=0pt, %
            linecolor=nbframe-border, %
            linewidth=1pt, %
            usetwoside=false, %
            everyline=true, %
            innerlinewidth=3pt, %
            innerlinecolor=nbframe-bg, %
            middlelinewidth=1pt, %
            middlelinecolor=nbframe-bg, %
            outerlinewidth=0.5pt, %
            outerlinecolor=nbframe-border, %
            needspace=0pt
        ]}
        {\end{mdframed}}
        
        \newenvironment{InvisibleVerbatim}
        {\begin{mdframed}[leftmargin=0.1\linewidth,innerleftmargin=3pt,innerrightmargin=3pt, userdefinedwidth=1\linewidth, linewidth=0pt, linecolor=white, usetwoside=false]}
        {\end{mdframed}}

        \renewenvironment{Verbatim}[1][\unskip]
        {\begin{alltt}\smaller}
        {\end{alltt}}
    

    % Help prevent overflowing lines due to urls and other hard-to-break 
    % entities.  This doesn't catch everything...
    \sloppy

    % Document level variables
    \title{waveEquation}
    \date{October 04, 2013}
    \release{}
    \author{Milad H. Mobarhan \& Svenn-Arne Dragly}
    \renewcommand{\releasename}{}

    % TODO: Add option for the user to specify a logo for his/her export.
    \newcommand{\sphinxlogo}{}

    % Make the index page of the document.
    \makeindex

    % Import sphinx document type specifics.
     


% Body

    % Start of the document
    \begin{document}

        
            \maketitle
        

        


        
        \section{Project 2: 2D wave equation}

\emph{Summary.} The aim of this project is to develop a solver for the
two-dimensional, standard, linear wave equation, with damping, and
verify the solver.\begin{center}\rule{3in}{0.4pt}\end{center}

\subsection{Mathematical problem}

The general wave equation in $d$ space dimensions, with variable
coefficients, can be written in the compact form

\[
\varrho\frac{\partial^2 u}{\partial t^2} +b \frac{\partial u}{\partial t} = \nabla\cdot (q\nabla u) + f\hbox{ for }\boldsymbol{x}\in\Omega\subset\mathbb{R}^d,\ t\in (0,T],
\]

which in 2D becomes

\[
\varrho(x,y)
\frac{\partial^2 u}{\partial t^2} +b \frac{\partial u}{\partial t} =
\frac{\partial}{\partial x}\left( q(x,y)
\frac{\partial u}{\partial x}\right)
+
\frac{\partial}{\partial y}\left( q(x,y)
\frac{\partial u}{\partial y}\right)
+ f(x,y,t)
\thinspace .
\] To save some writing and space we may use the index notation, where
subscript $t$, $x$ or $y$ means differentiation with respect to that
coordinate, i.e.

\begin{align*}
u_{t} &=\frac{\partial u}{\partial t} ,
\quad u_{tt} =\frac{\partial^2 u}{\partial t^2}\\
u_{x} &=\frac{\partial u}{\partial x} ,
\quad u_{xx} =\frac{\partial^2 u}{\partial x^2}\\
u_{y} &=\frac{\partial u}{\partial t} ,
\quad u_{yy} =\frac{\partial^2 u}{\partial y^2}\\
(q u_x)_x &= \frac{\partial}{\partial x}\left( q(x,y)
\frac{\partial u}{\partial x}\right)
\\
(q u_y)_y\ &=\frac{\partial}{\partial y}\left( q(x,y)
\frac{\partial u}{\partial y}\right)
\thinspace .
\end{align*}

The 2D versions of the two model PDEs, with and without variable
coefficients, can with now with the aid of the index notation for
differentiation be stated as \[
\varrho u_{tt} + b u_{t}= (q u_x)_x + (q u_z)_z + (q u_z)_z + f
\] Since this PDE contains a second-order derivative in time, we need
two initial conditions; (1) the initial shape of the string, $I$, and
(2) the initial velocity of the string, $V$;

\begin{align*}
u(x,y,0)&=I(x,y),\\
u_t(x,y,0)&=V(x,y).
\end{align*}

In addition, PDEs need boundary conditions, which are of three principal
types:

\begin{itemize}
\itemsep1pt\parskip0pt\parsep0pt
\item
  $u$ is prescribed ($u=0$ or a known time variation for an incoming
  wave)
\item
  $\partial u/\partial n = \boldsymbol{n}\cdot\nabla u$ prescribed (zero
  for reflecting boundaries)
\item
  An open boundary condition (also called radiation condition) is
  specified to let waves travel undisturbed out of the domain.
\end{itemize}\begin{center}\rule{3in}{0.4pt}\end{center}

\subsection{Discretization}

In this section we will derive the discrete set of equations to be
implemented in a program. We will for simplicity assume constant spacing
between the mesh points. Our mesh points are

\[
x_i = i\Delta x,\ i=0,\ldots,N_x,\quad \\ 
y_j = j\Delta y,\ j=0,\ldots,N_y,\quad \\
t_i = n\Delta t,\ n=0,\ldots,N_t{\thinspace .}
\] The solution $u(x,y,t)$ is sought at the mesh points. We introduce
the mesh function $u_{i,j}^n$, which approximates the exact solution at
the mesh point $(x_i,y_j,t_n)$ for $i=0,\ldots,N_x$, $j=0,\ldots,N_y$
and $n=0,\ldots,N_t$.\subsubsection{Discretizing the variable coefficient}

The principal idea is to first discretize the outer derivative. We
define

\[
\phi^x = q(x,y)
\frac{\partial u}{\partial x},
\] and use a centered derivative around $x=x_i$ for the derivative of
$\phi$:

\[
\left[\frac{\partial\phi^x}{\partial x}\right]^n_i
\approx \frac{\phi^x_{i+\frac{1}{2}} - \phi^x_{i-\frac{1}{2}}}{\Delta x}
= [D_x\phi^x]^n_i
\thinspace .
\] Then discretize

\[
\phi^x_{i+\frac{1}{2}}  = q_{i+\frac{1}{2}}
\left[\frac{\partial u}{\partial x}\right]^n_{i+\frac{1}{2}}
\approx q_{i+\frac{1}{2}} \frac{u^n_{i+1} - u^n_{i}}{\Delta x}
= [q D_x u]_{i+\frac{1}{2}}^n
\thinspace .
\] Similarly,

\[
\phi^x_{i-\frac{1}{2}}  = q_{i-\frac{1}{2}}
\left[\frac{\partial u}{\partial x}\right]^n_{i-\frac{1}{2}}
\approx q_{i-\frac{1}{2}} \frac{u^n_{i} - u^n_{i-1}}{\Delta x}
= [q D_x u]_{i-\frac{1}{2}}^n
\thinspace .
\] These intermediate results are now combined to

\[
\left[
     \frac{\partial}{\partial x}\left( q
     \frac{\partial u}{\partial x}\right)\right]^n_i
     \approx \frac{1}{\Delta x^2}
     \left( q_{i+\frac{1}{2}} \left({u^n_{i+1} - u^n_{i}}\right)
     - q_{i-\frac{1}{2}} \left({u^n_{i} - u^n_{i-1}}\right)\right)
\] With operator notation we can write the discretization as

\[
\left[
     \frac{\partial}{\partial x}\left( q
     \frac{\partial u}{\partial x}\right)\right]^n_i
     \approx [D_xq D_x u]^n_i
\]

Similarly we have

\[
\left[
     \frac{\partial}{\partial y}\left( q
     \frac{\partial u}{\partial y}\right)\right]^n_i
     \approx [D_yq D_y u]^n_i
\] for \[
\phi^y = q(x,y)
\frac{\partial u}{\partial y}.
\]

In order to compute $[D_xq D_x u]^n_i$ and $[D_yq D_y u]^n_i$ we need to
evaluate $q_{i\pm\frac{1}{2}}$. If $q$ is a known function, we can
easily evaluate $q_{i\pm\frac{1}{2}}$ simply as $q(x_{i\pm\frac{1}{2}})$
with $x_{i+\frac{1}{2}} = x_i + \frac{1}{2}\Delta x$. However, in many
cases $q$, is only known as a discrete function, and evaluating must be
done by averaging. The most commonly used averaging technique is the
arithmetic mean:

\[
q_{i+\frac{1}{2}} \approx
     \frac{1}{2}\left(q_{i+1}+ q_{i}\right) =
     [\overline{q}^{x}]_{i+\frac{1}{2}},\\
q_{i-\frac{1}{2}} \approx
     \frac{1}{2}\left(q_{i}+ q_{i-1}\right) =
     [\overline{q}^{x}]_{i-\frac{1}{2}},  
\] which we are going to use in the following.\subsubsection{Replacing derivatives by finite differences}

The second-order derivatives can be replaced by central differences. The
most widely used difference approximation of the second-order derivative
is

\[
\frac{\partial^2}{\partial t^2}u(x_i,y_j,t_n)\approx
\frac{u_{i,j}^{n+1} - 2u_{i,j}^n + u^{n-1}_{i,j}}{\Delta t^2}{\thinspace .}
\] It is convenient to introduce the finite difference operator notation
\[
[D_tD_t u]^n_{i,j} = \frac{u_{i,j}^{n+1} - 2u_{i,j}^n + u^{n-1}_{i,j}}{\Delta t^2}{\thinspace .}
\]

The first-order derivative in time in the damping term can be
approximated by

\[
[D_{2t} u]^n_{i,j} = \frac{u_{i,j}^{n+1} - u^{n-1}_{i,j}}  {2\Delta t}{\thinspace .}
\]\subsubsection{Algebraic version of the initial conditions}

The initial conditions are given as

\begin{align*}
u(x,y,t_0)&=I(x,y),\\
u_t(x,y,t_0)&=V(x,y).
\end{align*}

The first condition can be computed by \[
u_i^0 = I(x_i),\quad i=0,\ldots,N_x{\thinspace .}
\] For the second one we use a centered difference of type \[
\frac{\partial}{\partial t} u(x_i,y_j,t_0)\approx
\frac{u^1_{i,j} - u^{-1}_{i,j}}{2\Delta t} = [D_{2t} u]^0_{i,j} = V(x_i,y_j).
\]\subsubsection{Algebraic version of the PDE}

\paragraph{Interior spatial mesh points}

The PDE with variable coefficients is discretized term by term:

\[
[\varrho D_tD_t u + b D_{2t} u  = (D_x\overline{q}^x D_x u +
D_y\overline{q}^y D_yu ) + f]^n_{i,j}
\thinspace .
\] When written out and solved for the unknown $u^{n+1}_{i,j}$ one gets
the scheme

\begin{align*}u^{n+1}_{i,j} &= 
\Bigg\{
2u^{n}_{i,j} + u^{n-1}_{i,j}\left[\frac{b \Delta t}{2\varrho_{i,j}} - 1\right]+\frac{\Delta t^2}{\varrho_{i,j}} f^n_{i,j} \\
&+ \frac{1}{\varrho_{i,j}}\frac{\Delta t^2}{\Delta x^2} 
\left[ \frac{1}{2}(q_{i,j} + q_{i+1,j})(u^{n}_{i+1,j} - u^{n}_{i,j}) - \frac{1}{2}(q_{i-1,j} + q_{i,j})(u^{n}_{i,j} - u^{n}_{i-1,j})\right] \\
&+ \frac{1}{\varrho_{i,j}}\frac{\Delta t^2}{\Delta y^2} 
\left[ \frac{1}{2}(q_{i,j} + q_{i,j+1})(u^{n}_{i,j+1} - u^{n}_{i,j}) -\frac{1}{2}(q_{i,j-1} + q_{i,j})(u^{n}_{i,j} - u^{n}_{i,j-1})\right]
\Bigg\} \left(\frac{1}{1+ \frac{b\Delta t}{2\varrho_{i,j} }}\right )
\thinspace .\end{align*}\paragraph{Modified scheme for the first step}

A problem with algebraic version of the PDE equation arises when $n=0$
since the formula for $u^1_{i,j}$ involves $u^{-1}_{i,j}$, which is an
undefined quantity outside the time mesh (and the time domain). However,
we can use the initial condition to arrive at a special formula for
$u^1_{i,j}$. From initial condition we have

\[
 u^{-1}_{i,j} =u^1_{i,j} -  2\Delta tV.
\] Inserting this into the algebraic version of the PDE equation for
$n=0$ gives:

\begin{align*}u^{1}_{i,j} &= 
\Bigg\{
2u^{0}_{i,j} + \left( u^1_{i,j} -  2\Delta tV\right)\left[\frac{b \Delta t}{2\varrho_{i,j}} - 1\right]+\frac{\Delta t^2}{\varrho_{i,j}} f^0_{i,j} \\
&+ \frac{1}{\varrho_{i,j}}\frac{\Delta t^2}{\Delta x^2} 
\left[ \frac{1}{2}(q_{i,j} + q_{i+1,j})(u^{0}_{i+1,j} - u^{0}_{i,j}) - \frac{1}{2}(q_{i-1,j} + q_{i,j})(u^{0}_{i,j} - u^{0}_{i-1,j})\right] \\
&+ \frac{1}{\varrho_{i,j}}\frac{\Delta t^2}{\Delta y^2} 
\left[ \frac{1}{2}(q_{i,j} + q_{i,j+1})(u^{0}_{i,j+1} - u^{0}_{i,j}) -\frac{1}{2}(q_{i,j-1} + q_{i,j})(u^{0}_{i,j} - u^{0}_{i,j-1})\right]
\Bigg\} \left(\frac{1}{1+ \frac{b\Delta t}{2\varrho_{i,j} }}\right )
\thinspace .\end{align*}

Thus, the special formula for $u^1_{i,j}$ is

\begin{align*}
u^{1}_{i,j} &= 
\Bigg\{
2u^{0}_{i,j}  -  2\Delta tV\left[\frac{b \Delta t}{2\varrho_{i,j}} - 1\right]+\frac{\Delta t^2}{\varrho_{i,j}} f^0_{i,j} \\
&+ \frac{1}{\varrho_{i,j}}\frac{\Delta t^2}{\Delta x^2} 
\left[ \frac{1}{2}(q_{i,j} + q_{i+1,j})(u^{0}_{i+1,j} - u^{0}_{i,j}) - \frac{1}{2}(q_{i-1,j} + q_{i,j})(u^{0}_{i,j} - u^{0}_{i-1,j})\right] \\
&+ \frac{1}{\varrho_{i,j}}\frac{\Delta t^2}{\Delta y^2} 
\left[ \frac{1}{2}(q_{i,j} + q_{i,j+1})(u^{0}_{i,j+1} - u^{0}_{i,j}) -\frac{1}{2}(q_{i,j-1} + q_{i,j})(u^{0}_{i,j} - u^{0}_{i,j-1})\right]
\Bigg\} \left(\frac{1}{1+ \frac{b\Delta t}{2\varrho_{i,j} }}\right )\left(\frac{1}{2-\frac{b \Delta t}{2\varrho_{i,j}}} \right)
\thinspace .\end{align*}\begin{center}\rule{3in}{0.4pt}\end{center}

\subsection{Boundary condition}

In a rectangular spatial domain $\Omega = [0,L_x]\times [0,L_y]$, the
homogeneous Dirichlet condition, is given by

\[
u_{i,j}^{1} = u_{i,j}^{n+1} = 0 
\] for $n=1,2,\ldots,N_t-1$, when $i=0, \; i=N_x,\; j \in [0,L_y]$ and
when $j=0,\; j=N_y,\; i \in [0,L_x]$. Note that $u_{i,j}^{0}$ is given
by $I$, i.e.

\[
u_{i,j}^{0} = I(x_i,y_j) \quad \hbox{for   } i=0,\ldots,N_x \quad j=0,\ldots,Ny
\]

The Neumann boundary condition is given by

\begin{equation}
 \frac{\partial u}{\partial x}= \frac{\partial u}{\partial y} = 0,
\end{equation}

at $x = 0, N_x$ and $y = 0, N_y$. Since we have used central differences
in all the other approximations to derivatives in the scheme, it is
tempting to implement this condition by the difference

\begin{align*}
[-D_{2x} u = 0]^n_{0,j},\quad[D_{2x} u = 0]^n_{L_x,j}
&\Rightarrow\quad u^n_{-1,j}=u^n_{1,j}, \quad u^n_{N_x+1,j}=u^n_{N_x-1,j} \\
[-D_{2y} u = 0]^n_{i,0},\quad[D_{2y} u = 0]^n_{i,L_y}
&\Rightarrow\quad u^n_{i,-1}=u^n_{i,1}, \quad u^n_{i,N_y+1}=u^n_{i,N_y-1}
\thinspace .
\end{align*}

The problem is that $u^n_{-1,j}$, $u^n_{i,-1}$, $u^n_{N_x+1,j}$ and is
not a u value that is being computed since the point is outside the
mesh.

    % Make sure that atleast 4 lines are below the HR
    \needspace{4\baselineskip}

    
        \vspace{6pt}
        \makebox[0.1\linewidth]{\smaller\hfill\tt\color{nbframe-in-prompt}In\hspace{4pt}{[}{]}:\hspace{4pt}}\\*
        \vspace{-2.65\baselineskip}
        \begin{ColorVerbatim}
            \vspace{-0.7\baselineskip}
            \begin{Verbatim}[commandchars=\\\{\}]

\end{Verbatim}

            
                \vspace{0.3\baselineskip}
            
        \end{ColorVerbatim}
    

        

        \renewcommand{\indexname}{Index}
        \printindex

    % End of document
    \end{document}


